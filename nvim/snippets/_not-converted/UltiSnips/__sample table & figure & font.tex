% For table
%%%%%%%%%%%%%%%%%%%%%%%%%%%%%%%%%%%%%%%%%%%%%%%%%%%%%%%%%%%%%%%%%%%%%%%%

% \multicolumn{2}{|c|}{b} & c \\ \hline
% tabular 대신 \usepackage{longtable} 가능
\begin{longtable}[l or c or r]{l r l}  식으로
\end{longtable}

\begin{table}
	\centering
	\caption{計算結果}
	\label{tab:rst2}
	\begin{tabular}{l l}
		\toprule
		 & \\
		\midrule
		 & \\
		\bottomrule
	\end{tabular}
\end{table}

% file name should not include . inside
\begin{figure} % option htbp / ! for force it
	\centering
	% natheight, natwidth for bounding box
	\includegraphics[width=0.6\textwidth]{}
	\caption{}
	\label{}
\end{figure}

\begin{figure}
	\centering
	\begin{subfigure}[b]{0.3\textwidth}
		\includegraphics[width=\textwidth]{}
		\caption{}
		\label{fig:}
	\end{subfigure}
	~ %add desired spacing between images, e. g. ~, \quad, \qquad, \hfill etc. 
	%(or a blank line to force the subfigure onto a new line)
	\begin{subfigure}[b]{0.3\textwidth}
		\includegraphics[width=\textwidth]{}
		\caption{}
		\label{fig:}
	\end{subfigure}
	~ %add desired spacing between images, e. g. ~, \quad, \qquad, \hfill etc. 
	%(or a blank line to force the subfigure onto a new line)
	\begin{subfigure}[b]{0.3\textwidth}
		\includegraphics[width=\textwidth]{}
		\caption{}
		\label{fig:}
	\end{subfigure}
	\caption{}\label{fig:}
\end{figure}

% full width space :     
\paragraph{} \\
\subparagraph{} \\

% font
%%%%%%%%%%%%%%%%%%%%%%%%%%%%%%%%%%%%%%%%%%%%%%%%%%%%%%%%%%%%%%%%%%%%%%%%wkj
\underline{}
\emph{emph}
\textit{type italic} % italic
\texttt{typewriter} %type
\textsf{sans serif} % sans serif
\textbf{bold}
